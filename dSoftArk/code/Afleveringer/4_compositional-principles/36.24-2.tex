The main issue of using multiple inheritance to implement the semiciv variant,
is that we'll experience, what is known as 'the diamond problem' that is, the
inheritance UML diagram forms a diamond, and as such, the semiciv would have
multiple copies of it's super.super class, this is an issue, because we'll have
to keep all of these objects in sync.
Beside of this issue is also the issue of which super's function to call, an
issue that can be resolved in several ways.
In C++ this is handled with what's called virtual inheritance.

However generally this is just a way to avoid an issue, that one really
shouldn't get into, as the need for the UML diamond is a desperate warning of a
possibly bad design. 

But the benefits of this, compared to the approach we decided to use when we
implemented our semiciv design, is that in our design there's actually some
duplicate code, and as such, if there's a change to the way that betaciv loads
it's strategies, we'll have to update our semiciv strategy accordingly.
Whereas in the inheritance approach, it is automatically updated to reflect this change.

Do note however that some languages (for instance lua), can handle this multiple
inheritance in a somewhat prettier way, using a multiple inheritance control
function, which is basically a function that determins how the multiple
inheritance should be performed.
That is if there should be multiple super-super classes, or if there should be
a shared one, or even a hybrid wheres some super classes share super-supers in
groups.
With this function you can also decide which functions are to be overloaded,
that is which ones are obtainable and which one is the default. 
However this is mainly possibly because Lua is a dynamically typed language, and
because you implement inheritance youself in Lua using the metatable.
Hence you have full control as to how.
