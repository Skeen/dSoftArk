\subsubsection*{The software unit that defines the world layout for AlphaCiv is a test stub.}
Such is a statement, but is it true? 

\subparagraph*{ The definition of test stubs: }
A Test Stop is a replacement of a real \textsl{depended-on unit} that feeds indirect input, defined by the test code, into the \textsl{unit under test}.

Alternatively: Test specific software units that have the same interface but van be programmed by the testing code to provide specific indirect input.
This makes automated testing feasible. 

\subparagraph*{ The properties of test stubs: }
\begin{itemize} \itemsep1pt \parskip0pt \parsep0pt
\item ... getting the world under control 
\item Test Stubs make software testable.  
\item Get indirect input under control
\end{itemize}

\subsubsection*{ Conclusion about the statement: }

The world generated by AlphaCiv is a test stub when we realise that we generated it specifically to be able to test some features.

We test the UUT:Game\footnote{UUT stands for Unit Under Test. In this case Game is this UUT.} and we have introduced AlphaCiv as a test stub to simulate the DOU:GameWorld\footnote{DOU stands for Dependend Upon Unit.}. 

\subparagraph*{Other examples of test stubs:}
epsilonciv.TestWorld and epsilonciv.Test-StrategyFactory.defaultWithExtension.

However these are in the test-folder, as we have always seen this code as seperate from the specifications of our variants, whereas DefaultWorldGenStrategy\footnote{The software unit that implements the interface to generate the AlphaCiv World.} was a specification. 

So the statement is false, when we believe that the AlphaWorld is a part of the specifications, because then we can see DefaultWorldGenStrategy to be the UUT.
