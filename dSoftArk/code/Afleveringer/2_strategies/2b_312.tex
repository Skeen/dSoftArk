%  3-1-2 process for one of the variants. For one of them, discuss the use of the 3-1-2 process and the Strategy pattern. 

\subsubsection*{ \raisebox{.5pt}{\textcircled{\raisebox{-.9pt} {3}}} - Identifying behavior that varies}

In betaciv we first encountered a need for a different demand set for winning.
The new demand set was a need for a single player to own all existing cities. 
This is a demand set that is likely to change. 
Fx could another demand be that the players must vote every 4 rounds to possible decide a game winner, and only by majority vote can a winner be found.


\subsubsection*{ \raisebox{.5pt}{\textcircled{\raisebox{-.9pt} {1}}} - Stating the responsibility in an interface}

The only place the functionality differs, is in Game.getWinner(), and we rewrite the method like this:

\begin{lstlisting}
public Player getWinner()
{
    return winStrategy.getWinner();
}
\end{lstlisting}

Then we need the interface to be like this:
\begin{lstlisting}
public interface WinStrategy
{
    public Player getWinner();
}
\end{lstlisting}

\subsubsection*{ \raisebox{.5pt}{\textcircled{\raisebox{-.9pt} {2}}} - Composing the behavior by delegation}

In the attempt to compose the delegated behavior, we stumble upon a problem.
The behavior needs to know about part of the state of the game, namely the age, and in the case of the new variant, also the ownership of the cities.
They are accessible through the method Game.getAge() and Game.getCityAt().getOwner().

We chose to seed the game-state to the interface as a parameter to getWinner().
The interface looks like this now:
\lstinputlisting[linerange=5-14]{../../hotciv/src/hotciv/framework/strategy/WinStrategy.java}

