As we follow the TDD approach,
we have kept an allways up-to-date log of our work, specifying which steps in the TDD iteration process we have processed and are currently at,
together with relevant focus-areas.


\subsection*{TDD Rythm and Principles}
This section includes excerpts from our development log. The numbers to the left specifies the line numbers in our log.

The number in the file in the range of [0,5] specifies the step in the TDD iteration\footnote{See [FRS] on TDD Rhythm}.
A step 6 is implicit in all iterations. It means that the refactorizations in step 5 is tested and do not introduce errors.

% >> Archer type unit at 2.0
\logInput{204}{218}

This is an interesting iteration because it shows a deviation from the (TDD) Rhytm:

We do not succesfully implement some changes in our step 3 to be able to proceed to step 4.

Yet failing the test is also a step in the Rhythm, and we just proceed as if we are at the correct step 2 (line 209).

This is an example supporting the fact that the Rhythm always points to a next step, no matter the status of a Green Bar, and as such gives us progression and as a rule our overall process will allways evolve constructively when we use the Rhythm.

The iteration is also interesting, because until this iteration, we hadn't found much use of step 5, refactorization.

Of course refactorization requires enough 'fake it' or duplicate code, to be relevant and these only comes after some progression in the code, i.e. after some simple iterations. \\

The iteration is a followup of earlier iterations, where other units is expected found at other initial placements, (and we faked it earlier, so we didn't implement a full solution).
So this is the step were a triangulation forces us to not only fake it, but rewrite code, to accept a more abstract situation.
In this triangulation, we introduced variant handling with a simple parameterized solution.
This is implemented as the most conservative way of adding a layer of abstraction to the code to pass the test.
And as such we follow the Triangulation Principle. \\




% >> assert that, a red unit, that moves ontop of a blue unit, kills it <<
\logInput{312}{325}
This is also an interesting iteration.

This iteration is an example of the confidence the TDD approach has given us:
We trusted our tests so we weren't afraid to make changes and refactor our code.
We'll mention this later in the section about benefits.




%\subsection*{Refactoring Process}
